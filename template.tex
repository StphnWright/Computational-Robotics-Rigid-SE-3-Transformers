% template.tex
% COMS 4733 course staff, Columbia University
% (c) 2020

\documentclass[12pt,letterpaper]{article}
\usepackage[fleqn]{amsmath}
\makeatletter
\renewcommand*\env@matrix[1][*\c@MaxMatrixCols c]{%
  \hskip -\arraycolsep
  \let\@ifnextchar\new@ifnextchar
  \array{#1}}
\makeatother

\usepackage{amsthm}
\usepackage{amsfonts}
\usepackage{amssymb}
\usepackage{amscd}
\usepackage{enumerate}
\usepackage{fancyhdr}
\usepackage{mathrsfs}
\usepackage{bbm}
\usepackage{framed}
\usepackage{mdframed}
\usepackage{listings}
\usepackage{cancel}
\usepackage{mathtools}
\usepackage{verbatim}
\usepackage{enumitem}
\usepackage[letterpaper,voffset=-.5in,bmargin=3cm,footskip=1cm]{geometry}
\usepackage[colorlinks = true]{hyperref} 
\setlength{\parindent}{0.0in}
\setlength{\parskip}{0.1in}
\allowdisplaybreaks
\headheight 15pt
\headsep 10pt
\newcommand\N{\mathbb N}
\newcommand\Z{\mathbb Z}
\newcommand\R{\mathbb R}
\newcommand\Q{\mathbb Q}
\newcommand\lcm{\operatorname{lcm}}
\newcommand\setbuilder[2]{\ensuremath{\left\{#1\;\middle|\;#2\right\}}}
\newcommand\E{\operatorname{E}}
\newcommand\V{\operatorname{V}}
\newcommand\Pow{\ensuremath{\operatorname{\mathcal{P}}}}

\newcommand{\rca}{\cos\alpha} \newcommand{\rcb}{\cos\beta} \newcommand{\rcg}{\cos\gamma}
\newcommand{\rsa}{\sin\alpha} \newcommand{\rsb}{\sin\beta} \newcommand{\rsg}{\sin\gamma}
\newcommand{\degsym}{\ensuremath{^{\circ}}}

\DeclarePairedDelimiter\ceil{\lceil}{\rceil}
\DeclarePairedDelimiter\floor{\lfloor}{\rfloor}
\newcommand\hint[1]{\textbf{Hint}: #1}
\newcommand\note[1]{\textbf{Note}: #1}

\lstset{
  basicstyle=\ttfamily,
  columns=fullflexible,
  frame=single,
  breaklines=true,
  postbreak=\mbox{\textcolor{red}{$\hookrightarrow$}},
}

\fancypagestyle{firstpagestyle} {
  \renewcommand{\headrulewidth}{0pt}
  \lhead{\textbf{COMS 4733}}
  \chead{\textbf{Stephen V. Wright}}
  \rhead{SVW2112}
}

\pagestyle{fancyplain}
\usepackage{tikz}
\hbadness=10000
\begin{document}
  \thispagestyle{firstpagestyle}
  \begin{center}
    {\huge \textbf{Homework 0}}
  \end{center}


    No special notes to consider for grading.
    
    

\subsection*{Problem 1}

\begin{enumerate}[leftmargin=*]
\item Answer: $B$ and $D$ belong to the $SE(3)$ group of valid transforms, but $A$ and $C$ do \textit{not} belong to $SE(3)$ group of valid transforms. See the following justifications for each matrix.\\

\textbf{Matrix A}\\
$A$ does not have the form $\begin{bmatrix} R & r \\ 0 & 1 \end{bmatrix}$ where $R \in {\R}^{3x3}$, $r \in {\R}^3$. Therefore, $A \notin SE(3)$.\\

The remaining matrices $B$, $C$, and $D$ do have the required form above.\\

\textbf{Matrix B}\\

$R = \begin{bmatrix} 1 & 0 & 0\\ 0 & 0 & -1\\ 0 & 1 & 0\\ \end{bmatrix}$\\

$\det(R) = (1) \det \begin{bmatrix} 0 & -1\\ 1 & 0\\ \end{bmatrix} = 0(0)-1(-1) = 1$\\

$R^{T} R = 
\begin{bmatrix} 1 & 0 & 0\\ 0 & 0 & 1\\ 0 & -1 & 0\\ \end{bmatrix} 
\begin{bmatrix} 1 & 0 & 0\\ 0 & 0 & -1\\ 0 & 1 & 0\\ \end{bmatrix}  
= \begin{bmatrix} 1 & 0 & 0\\ 0 & 1 & 0\\ 0 & 0 & 1\\ \end{bmatrix} 
= I$\\

$R R^{T} = 
\begin{bmatrix} 1 & 0 & 0\\ 0 & 0 & -1\\ 0 & 1 & 0\\ \end{bmatrix} 
\begin{bmatrix} 1 & 0 & 0\\ 0 & 0 & 1\\ 0 & -1 & 0\\ \end{bmatrix}
= \begin{bmatrix} 1 & 0 & 0\\ 0 & 1 & 0\\ 0 & 0 & 1\\ \end{bmatrix}
=I$\\

Therefore, $B \in SE(3)$.\\

\textbf{Matrix C}\\

$R = \begin{bmatrix} 1 & 0 & 0\\ 0 & 0 & -1\\ 0 & -1 & 0\\ \end{bmatrix}$\\

$\det(R) = (1) \det \begin{bmatrix} 0 & -1\\ -1 & 0\\ \end{bmatrix} = 0(0)-(-1)(-1) = -1 \neq 1$\\

Therefore, $C \notin SE(3)$.\\

\textbf{Matrix D}\\

$R = \begin{bmatrix} -1 & 0 & 0\\ 0 & 0 & 1\\ 0 & 1 & 0\\ \end{bmatrix}$\\

$\det(R) = (-1) \det \begin{bmatrix} 0 & 1\\ 1 & 0\\ \end{bmatrix} = -(0(0)-1(1)) = 1$\\

$R^{T} R = 
\begin{bmatrix} -1 & 0 & 0\\ 0 & 0 & 1\\ 0 & 1 & 0\\ \end{bmatrix} 
\begin{bmatrix} -1 & 0 & 0\\ 0 & 0 & 1\\ 0 & 1 & 0\\ \end{bmatrix}  
= \begin{bmatrix} 1 & 0 & 0\\ 0 & 1 & 0\\ 0 & 0 & 1\\ \end{bmatrix} 
= I$\\

$R R^{T} = 
\begin{bmatrix} 1 & 0 & 0\\ 0 & 0 & -1\\ 0 & 1 & 0\\ \end{bmatrix} 
\begin{bmatrix} -1 & 0 & 0\\ 0 & 0 & 1\\ 0 & 1 & 0\\ \end{bmatrix}
= \begin{bmatrix} 1 & 0 & 0\\ 0 & 1 & 0\\ 0 & 0 & 1\\ \end{bmatrix}
=I$\\

Therefore, $D \in SE(3)$.\\

\item $SE(3) \cap \{ A, B, C, D \} = \{ B, D \} $\\

\textbf{Matrix B}

\vspace{-2em}
\begin{equation*}
\hspace{-2.5em}\begin{split}
[B | I] 
& = \begin{bmatrix}[cccc|cccc]
1 & 0 & 0  & 1 & 1 & 0 & 0 & 0\\
0 & 0 & -1 & 3 & 0 & 1 & 0 & 0\\
0 & 1 & 0  & 5 & 0 & 0 & 1 & 0\\
0 & 0 & 0  & 1 & 0 & 0 & 0 & 1\\
\end{bmatrix} R_2 \leftrightarrow -R_3 &\\
& = \begin{bmatrix}[cccc|cccc]
1 & 0 & 0 & 1  & 1 & 0  & 0 & 0\\
0 & 1 & 0 & 5  & 0 & 0  & 1 & 0\\
0 & 0 & 1 & -3 & 0 & -1 & 1 & 0\\
0 & 0 & 0 & 1  & 0 & 0  & 0 & 1\\
\end{bmatrix} \begin{aligned} R_1 \rightarrow R_1 - R_4 \\ R_2 \rightarrow R_2 - 5 R_4 \\ R_3 \rightarrow R_3 + 3 R_4 \end{aligned} &\\
& = \begin{bmatrix}[cccc|cccc]
1 & 0 & 0 & 0 & 1 & 0  & 0 & -1\\
0 & 1 & 0 & 0 & 0 & 0  & 1 & -5\\
0 & 0 & 1 & 0 & 0 & -1 & 0 &  3\\
0 & 0 & 0 & 1 & 0 & 0  & 0 &  1\\
\end{bmatrix} = [I | B^{-1}] 
\end{split}
\end{equation*}

$\Rightarrow B^{-1} = \begin{bmatrix}
1 & 0  & 0 & -1\\
0 & 0  & 1 & -5\\
0 & -1 & 0 &  3\\
0 & 0  & 0 &  1\\
\end{bmatrix}$

$B^{-1} B = 
\begin{bmatrix}
1 & 0  & 0 & -1\\
0 & 0  & 1 & -5\\
0 & -1 & 0 &  3\\
0 & 0  & 0 &  1\\
\end{bmatrix}
\begin{bmatrix}
1 & 0  &  0 & 1\\
0 & 0  & -1 & 3\\
0 & 1 &  0 & 5\\
0 & 0  &  0 & 1\\
\end{bmatrix}
= \begin{bmatrix} 1 & 0 & 0 & 0\\ 0 & 1 & 0 & 0\\ 0 & 0 & 1 & 0\\ 0 & 0 & 0 & 1\\ \end{bmatrix} 
= I$\\

$B B^{-1} = 
\begin{bmatrix}
1 & 0  &  0 & 1\\
0 & 0  & -1 & 3\\
0 & 1 &  0 & 5\\
0 & 0  &  0 & 1\\
\end{bmatrix}
\begin{bmatrix}
1 & 0  & 0 & -1\\
0 & 0  & 1 & -5\\
0 & -1 & 0 &  3\\
0 & 0  & 0 &  1\\
\end{bmatrix}
= \begin{bmatrix} 1 & 0 & 0 & 0\\ 0 & 1 & 0 & 0\\ 0 & 0 & 1 & 0\\ 0 & 0 & 0 & 1\\ \end{bmatrix} 
= I$\\

\textbf{Matrix D}

\vspace{-2em}
\begin{equation*}
\hspace{-2.5em}\begin{split}
[D | I] 
& = \begin{bmatrix}[cccc|cccc]
-1 & 0 & 0 & 1 & 1 & 0 & 0 & 0\\
 0 & 0 & 1 & 3 & 0 & 1 & 0 & 0\\
 0 & 1 & 0 & 5 & 0 & 0 & 1 & 0\\
 0 & 0 & 0 & 1 & 0 & 0 & 0 & 1\\
\end{bmatrix} \begin{aligned} R_1 \rightarrow -R_1 \\ R_2 \leftrightarrow R_3 \end{aligned} &\\
& = \begin{bmatrix}[cccc|cccc]
1 & 0 & 0 & -1 & -1 & 0 & 0 & 0\\
0 & 1 & 0 &  5 &  0 & 0 & 1 & 0\\
0 & 0 & 1 &  3 &  0 & 1 & 0 & 0\\
0 & 0 & 0 &  1 &  0 & 0 & 0 & 1\\
\end{bmatrix} \begin{aligned} R_1 \rightarrow R_1 + R_4 \\ R_2 \rightarrow R_2 - 5 R_4 \\ R_3 \rightarrow R_3 - 3 R_4 \end{aligned} &\\
& = \begin{bmatrix}[cccc|cccc]
1 & 0 & 0 & 0 & -1 & 0 & 0 &  1\\
0 & 1 & 0 & 0 &  0 & 0 & 1 & -5\\
0 & 0 & 1 & 0 &  0 & 1 & 0 & -3\\
0 & 0 & 0 & 1 &  0 & 0 & 0 &  1\\
\end{bmatrix} = [I | D^{-1}] 
\end{split}
\end{equation*}

$\Rightarrow D^{-1} = \begin{bmatrix}
-1 & 0 & 0 &  1\\
 0 & 0 & 1 & -5\\
 0 & 1 & 0 & -3\\
 0 & 0 & 0 &  1\\
\end{bmatrix}$

$D^{-1} D = 
\begin{bmatrix}
-1 & 0 & 0 &  1\\
 0 & 0 & 1 & -5\\
 0 & 1 & 0 & -3\\
 0 & 0 & 0 &  1\\
\end{bmatrix}
\begin{bmatrix}
-1 & 0 & 0 & 1\\
 0 & 0 & 1 & 3\\
 0 & 1 & 0 & 5\\
 0 & 0 & 0 & 1\\
\end{bmatrix}
= \begin{bmatrix} 1 & 0 & 0 & 0\\ 0 & 1 & 0 & 0\\ 0 & 0 & 1 & 0\\ 0 & 0 & 0 & 1\\ \end{bmatrix} 
= I$\\

$D D^{-1} = 
\begin{bmatrix}
-1 & 0 & 0 & 1\\
 0 & 0 & 1 & 3\\
 0 & 1 & 0 & 5\\
 0 & 0 & 0 & 1\\
\end{bmatrix}
\begin{bmatrix}
-1 & 0 & 0 &  1\\
 0 & 0 & 1 & -5\\
 0 & 1 & 0 & -3\\
 0 & 0 & 0 &  1\\
\end{bmatrix}
= \begin{bmatrix} 1 & 0 & 0 & 0\\ 0 & 1 & 0 & 0\\ 0 & 0 & 1 & 0\\ 0 & 0 & 0 & 1\\ \end{bmatrix} 
= I$\\

\end{enumerate}

\subsection*{Problem 2}

\begin{enumerate}
\item 
$
\begin{bmatrix} ^{1}p \\ 1\end{bmatrix}
= ^{1}T_2 \begin{bmatrix} ^{2}p \\ 1\end{bmatrix}
= \begin{bmatrix}
 0 & 1 & 0 &   0\\
-1 & 0 & 0 &  -3\\
 0 & 0 & 1 & -10\\
 0 & 0 & 0 &   1\\
\end{bmatrix}
\begin{bmatrix} 2 \\ 4 \\ 6 \\ 1\end{bmatrix}
= \begin{bmatrix} -1(4) \\ -1(2) - 3(1) \\ 1(6) - 10(1) \\ 1(1) \end{bmatrix}
= \begin{bmatrix} 4 \\ -5 \\ -4 \\ 1\end{bmatrix}
$

Thus $^{1}p = \begin{bmatrix} 4 & -5 & -4\end{bmatrix}^T$.

\item 
$
^{0}T_2 = ^{0}T_1 ^{1}T_2
= \begin{bmatrix}
-1 & 0 & 0 &  0\\
 0 & 0 & 1 & 10\\
 0 & 1 & 0 &  3\\
 0 & 0 & 0 &  1\\
\end{bmatrix}
\begin{bmatrix}
 0 & 1 & 0 &   0\\
-1 & 0 & 0 &  -3\\
 0 & 0 & 1 & -10\\
 0 & 0 & 0 &   1\\
\end{bmatrix}
= \begin{bmatrix}
 0 & -1 & 0 & 0\\
 0 &  0 & 1 & 0\\
-1 &  0 & 0 & 0\\
 0 &  0 & 0 & 1\\
\end{bmatrix}
$.\\
Here $r = \begin{bmatrix} 0 \\ 0 \\ 0 \end{bmatrix}$, so there is no translation, and
$
R = \begin{bmatrix}
 0 & -1 & 0\\
 0 &  0 & 1\\
-1 &  0 & 0\\
\end{bmatrix}
$.


Match $R$ to the form
\vspace{-.5em}
\begin{equation*}
\begin{bmatrix}
\rca\rcb & \rca\rsb\rsg - \rsa\rcg & \rca\rsb\rsg + \rsa\rsg\\
\rsa\rcb & \rsa\rsb\rsg + \rca\rcg & \rsa\rsb\rsg - \rca\rsg\\
-\rsb & \rcb\rsg & \rcb\rsg \\
\end{bmatrix}
\end{equation*}

From entry $(3, 1)$, $-\rsb = -1 \Rightarrow \beta = 90\degsym$

Thus 
$R = 
\begin{bmatrix}
0 & -\rca\rsg - \rsa\rcg & -\rca\rsg + \rsa\rsg\\
0 & -\rsa\rsg + \rca\rcg & -\rsa\rsg - \rca\rsg\\
1 & 0 & 0 \\
\end{bmatrix}
$

Trying $\alpha = 90\degsym$ and $\gamma = 0$ matches the two matrices. Thus,  $\alpha = \beta = 90\degsym$ and $\gamma = 0$.\\

Therefore, this transform consists of rotations of $90\degsym$ about the $y$-axis and $90\degsym$ about the $z$-axis.

\end{enumerate}

\subsection*{Problem 3}

\begin{enumerate}

\item

\textbf{Pose $^{m}T_w$}\\
The steps in intrinsic coordinates are:
\begin{enumerate}
\item Rotate $+90\degsym$ about the $x$-axis of frame $m$.\\
Thus $\alpha = +90\degsym, \rsa = 1, \rcg = 0$ and 
$R_1 = \begin{bmatrix} 1 & 0 & 0 \\ 0 & 0 & -1 \\ 0 & 1 & 0 \end{bmatrix}$.
\item Rotate $-90\degsym$ about the $y$-axis of the previous frame.\\
Thus $\beta = -90\degsym, \rsb = -1, \rcb = 0$ and 
$R_2 = \begin{bmatrix} 0 & 0 & -1 \\ 0 & 1 & 0 \\ 1 & 0 & 0 \end{bmatrix}$.
\item In frame $m$, translate +4 in $x$ and +6 in $y$. Thus $r = \begin{bmatrix} 4 & 6 & 0 \end{bmatrix}^T$.
\end{enumerate}
Thus $R = R_1 R_2 = 
\begin{bmatrix} 1 & 0 & 0 \\ 0 & 0 & -1 \\ 0 & 1 & 0 \end{bmatrix}
\begin{bmatrix} 0 & 0 & -1 \\ 0 & 1 & 0 \\ 1 & 0 & 0 \end{bmatrix}
= \begin{bmatrix} 0 & 0 & -1 \\ -1 & 0 & 0 \\ 0 & 1 & 0 \end{bmatrix}
$ and \\
\vspace{.5em}
\begin{equation*}
\hspace{-2em}
^{m}T_w = \begin{bmatrix}
 0 & 0 & -1 & 4\\
-1 & 0 &  0 & 6\\
 0 & 1 &  0 & 0\\
 0 & 0 &  0 & 1\\
\end{bmatrix}.
\end{equation*}
\\
\textbf{Pose $^{e}T_m$}\\
The steps are:
\begin{enumerate}
\item Rotate $+90\degsym$ about the $y$-axis of frame $e$.\\
Thus $\beta = +90\degsym, \rsb = 1, \rcb = 0$ and 
$R = \begin{bmatrix} 0 & 0 & 1 \\ 0 & 1 & 0 \\ -1 & 0 & 0 \end{bmatrix}$.
\item In frame $e$, translate -10 in $x$. Thus $r = \begin{bmatrix} -10 & 0 & 0 \end{bmatrix}^T$.
\end{enumerate}
Thus $^{e}T_m = \begin{bmatrix}
 0 & 0 & 1 & -10\\
 0 & 1 & 0 &   0\\
-1 & 0 & 0 &   0\\
 0 & 0 & 0 &   1\\
\end{bmatrix}$.
\\
\textbf{Pose $^{e}T_w$}\\
The steps are:
\begin{enumerate}
\item Rotate $-90\degsym$ about the $z$-axis of frame $e$.\\
Thus $\alpha = -90\degsym, \rsa = -1, \rca = 0$ and 
$R = \begin{bmatrix} 0 & 1 & 0 \\ -1 & 0 & 0 \\ 0 & 0 & 1 \end{bmatrix}$.
\item In frame $e$, translate -10 in $x$, +6 in $y$, and -4 in $z$. Thus $r = \begin{bmatrix} -10 & 6 & 4 \end{bmatrix}^T$.
\end{enumerate}
Thus $^{e}T_w = \begin{bmatrix}
 0 & 1 & 0 & -10\\
-1 & 0 & 0 &   6\\
 0 & 0 & 1 &  -4\\
 0 & 0 & 0 &   1\\
\end{bmatrix}$.

\item
$
{^{e}T_m} {^{m}T_w}
= \begin{bmatrix}
 0 & 0 & 1 & -10\\
 0 & 1 & 0 &   0\\
-1 & 0 & 0 &   0\\
 0 & 0 & 0 &   1\\
\end{bmatrix}
\begin{bmatrix}
 0 & 0 & -1 & 4\\
-1 & 0 &  0 & 6\\
 0 & 1 &  0 & 0\\
 0 & 0 &  0 & 1\\
\end{bmatrix}
= \begin{bmatrix}
 0 & 1 & 0 & -10\\
-1 & 0 & 0 &   6\\
 0 & 0 & 1 &  -4\\
 0 & 0 & 0 &   1\\
\end{bmatrix}
= {^{e}T_w}
$.\\
\textbf{Pose $^{w}T_e$}\\
The steps are:
\begin{enumerate}
\item Rotate $+90\degsym$ about the $z$-axis of frame $w$.\\
Thus $\alpha = +90\degsym, \rsa = 1, \rca = 0$ and 
$R = \begin{bmatrix} 0 & -1 & 0 \\ 1 & 0 & 0 \\ 0 & 0 & 1 \end{bmatrix}$.
\item In frame $w$, translate +6 in $x$, +10 in $y$, and +4 in $z$.\\Thus $r = \begin{bmatrix} 6 & 10 & 4 \end{bmatrix}^T$.
\end{enumerate}
Thus $^{w}T_e = \begin{bmatrix}
0 & -1 & 0 &  6\\
1 &  0 & 0 & 10\\
0 &  0 & 1 &  4\\
0 &  0 & 0 &  1\\
\end{bmatrix}$.

${^{e}T_w} {^{w}T_e} = 
\begin{bmatrix}
 0 & 1 & 0 & -10\\
-1 & 0 & 0 &   6\\
 0 & 0 & 1 &  -4\\
 0 & 0 & 0 &   1\\
\end{bmatrix}
\begin{bmatrix}
0 & -1 & 0 &  6\\
1 &  0 & 0 & 10\\
0 &  0 & 1 &  4\\
0 &  0 & 0 &  1\\
\end{bmatrix}
= \begin{bmatrix} 1 & 0 & 0 & 0\\ 0 & 1 & 0 & 0\\ 0 & 0 & 1 & 0\\ 0 & 0 & 0 & 1\\ \end{bmatrix} 
= I$\\

${^{w}T_e} {^{e}T_w} = 
\begin{bmatrix}
0 & -1 & 0 &  6\\
1 &  0 & 0 & 10\\
0 &  0 & 1 &  4\\
0 &  0 & 0 &  1\\
\end{bmatrix}
\begin{bmatrix}
 0 & 1 & 0 & -10\\
-1 & 0 & 0 &   6\\
 0 & 0 & 1 &  -4\\
 0 & 0 & 0 &   1\\
\end{bmatrix}
= \begin{bmatrix} 1 & 0 & 0 & 0\\ 0 & 1 & 0 & 0\\ 0 & 0 & 1 & 0\\ 0 & 0 & 0 & 1\\ \end{bmatrix} 
= I$\\

By the definition of the matrix inverse, if $AB = BA = I$ for two square matrices $A$ and $B$, then $A$ and $B$ are inverses of each other, i.e., $B = A^{-1}$ and $A = B^{-1}$.\\

Given that ${^{e}T_w} {^{w}T_e} = {^{w}T_e} {^{e}T_w} = I$, it follows that ${^{e}T_w} = ({^{w}T_e})^{-1}$.

\end{enumerate}

\subsection*{Problem 4}

No additional comments.

\end{document}